\chapter{结论与展望}
\label{conclusion}
\section{结论}
对于机器人三维感知面临的巨大困难,本文基于RGB-D图像,通过引入深度学习,和传统算法相结合,旨在3D目标检测和位姿估计问题上有所突破,以推动整个3D视觉的发展,解决机器人的感知问题。本文主要有三个方面的贡献:第一,本文从3D视觉的深度信息的获取入手,对于当前3D相机性价比低的现状,提出了对偶RGB-D相机结构;第二,将深度学习和传统视觉算相结合,提出了3D-MRAI算法实现对目标物体的检测和位姿估计;第三,将3D-MRAI算法应用到实际机器人领域,解决Bin-Picking问题。

对偶RGB-D相机结构针对单个RGB-D相机采集的深度图缺失问题,通过增加一个旋转了180度的RGB-D相机与原相机构成对偶,然后结合两个相机的深度图以及通过两个相机彩色图构建的深度图来获取高质量的深度图,实验表明对偶RGB-D相机再结合了三张深度图后,生成的深度图比单个相机采集的深度图具有更高的的填充率,深度信息的精度也更高。

3D-MRAI算法针对3D目标检测和位姿估计问题的特点,分别对深度学习算法Faster/Mask R-CNN和传统视觉算法4PCS进行改进并结合,从而实现3D目标检测和位姿估计。实验表明,3D-MRAI在所采集的workpiece数据集上相比传统算法表现出了较高的检测准确率,估计的位姿精度也更高。

针对Bin-Picking问题,本文将对偶RGB-D结构与3D-MRAI算法实际应用,开发出的Bin-Picking视觉系统相比与传统Bin-Picking视觉系统,具有更高的抓取成功率、更快的响应速度,以及更低的成本。在所设计的抓取实验中达到了100\%的抓取成功率、约0.7s的响应时间。

\section{进一步工作的方向}
对于深度信息的获取,在不考虑成本的情况下,当然高价的3D相机获取的深度图质量高,本文使用两个低价的RGB-D相机构成对偶RGB-D相机也是无奈之举,低价RGB-D相机获取的深度信息在算法上再如何改进也无法对深度图质量产生质的改变,因此针对这一部分,还是寄希望于广大相机厂家推出高性价比的3D相机。

对于本文提出的3D-MRAI算法,进一步工作方向可以有以下三点:第一,3D-MRAI虽然能较好的检测目标并估计目标位姿,但其算法FPS只有2.2,对于实时性要求较高的情况如避障难以满足要求,因此进一步工作可以针对算法的运算时间进行改进,比如,可以通过对3D-MRAI中的深度网络进行裁剪,因为算法的运算时间主要耗费在深度神经网络的运算上。第二,3D-MRAI算法检测目标估计位姿分为两步:先检测出目标的BBox/Mask,然后再匹配3D模型和目标点云得到目标位姿。直觉上感觉不是那么完美,能否设计一个end-to-end的深度神经网络,一步计算出目标的位姿和种类。第三,3D-MRAI算法依赖于目标的3D模型,需要知道目标的3D模型才能得到目标位姿,大大限制了算法的实际应用,因此,进一步工作可以尝试解决3D模型未知情况下目标位姿的估计。

%%% Local Variables:
%%% TeX-master: "../thesis.tex"
%%% End: