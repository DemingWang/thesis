% 定义中英文摘要和关键字
\begin{cabstract}
  三维视觉是机器人感知的重要组成部分,但其目前的技术水平难以帮助机器人有效地感知周围的三维世界。随着近几年深度学习的发展,计算机视觉领域取得了巨大的发展,尤其是在2D视觉领域,2D目标的检测和分类的准确率得到了巨大的提升,但3D目标的检测并没有巨大的提升。因此,本文针对机器人目前三维感知的困难,通过参考深度学习在2D视觉上的突破,将其引入到3D视觉上来,提出了3D-MRAI算法,用于解决对3D目标的检测以及位姿的估计。

  深度信息的质量对3D目标检测以及位姿估计的准确率和精度都有至关重要的影响,因此为了获取高质量的深度信息,本文针对现有RGB-D相机的缺点,提出了对偶RGB-D相机结构,通过组合两个低价的RGB-D相机来获取高质量的深度信息,提高了相机获取深度图的填充率并且增强了深度信息的鲁棒性。

  为了能够在RGB-D图中检测出目标物体的种类以及位姿,本文提出了3D-MRAI算法,算法分为两步,第一步在相机拍摄的三维点云中分割出目标物体点云;第二步通过点云匹配算法求解出目标的位姿。为了分割出目标物体点云,本文设计了检测模块。所设计的检测模块在Mask R-CNN的基础上,通过将深度图变换为HHA图,有效地利用三维信息,并结合RGB图完成对目标物体的检测,解决了原算法难以检测缺少纹理物体的问题;检测模块还引入了Spatial Transformer Network结构以应对目标物体的各种姿态。为了求解出目标的位姿,本文设计了一个匹配模块,通过匹配模块匹配目标模型和目标点云来获得目标位姿。所设计的匹配模块在4PCS算法的基础上,通过滤去离群点、引入ICP算法提高了匹配精度。

  本文还将所提出的3D-MRAI算法实际应用到Bin-Picking问题上,设计了一个基于3D-MARI算法的随机分拣视觉系统,所设计的系统在实验中达到了100\%的抓取成功率,并且算法的运算时间也完全满足实际应用。
\end{cabstract}

\ckeywords{RGB-D,目标检测,位姿估计,随机分拣}


\begin{eabstract}
3D vision is an important part of robot perception, but the technology of 3D vision currently is hard to help robots effectively perceive the surrounding 3D world. With the development of deep learning in recent years, tremendous development has been made in the field of computer vision. Especially in the field of 2D vision, the accuracy of 2D object detection and classification has been greatly improved, but the detection of 3D objects has not been huge Enhance. Therefore, for the difficulty of current 3D robot perception. This paper proposed a new algorithm 3D-MRAI, which introduced deep learning into 3D vision based on the breakthrough of deep learning in 2D vision. This algorithm is proposed to solve the problem of 3D object detection and pose estimation .

High-quality depth map has a great influence on the results of 3D object detection and pose estimation. To acquire high-quality depth map, a dual RGB-D camera structure is proposed to overcome the shortcomings of the existing RGB-D cameras. Dual RGB-D camera can obtain high-quality depth map by combining two low-cost RGB-D cameras, which also increases the fill rate of depth map and enhances depth value in depth map.

3D-MRAI algorithm is proposed to detect object and estimate pose in a given RGB-D frame. 3D-MRAI runs in two stage. The first stage is to get the point cloud of the object from the RGB-D frame. The second stage is to estimate the object pose by point cloud matching. To segement the point cloud of the object, we designed a detection module. The detection module is based on Mask R-CNN. We convert depth map into HHA frame to take full advantage of depth value. By combining RGB and HHA, our detection module can detect textureless objects. Our detection module also  use Spatial Transformer Network to detect object in arbitrary pose.  To estimate the pose of object, a matching module is designed. We estimate the object's pose by matching the object model to object point cloud. The matching module is based on 4PCS. By filtering outliers and combining ICP algorithm, our matching module has higher accuracy.

  This paper also applies the proposed 3D-MRAI algorithm to solve the Bin-Picking problem. A bin-picking vision system is designed based on 3D-MARI algorithm. The designed system achieves 100\% successful picking rate, and the system's response time also fully meets the application.

\end{eabstract}

\ekeywords{RGB-D, object detection, pose estimation, Bin-Picking}

%%% Local Variables:
%%% TeX-master: "../thesis.tex"
%%% End: