% 定义中英文摘要和关键字
\begin{cabstract}
  三维视觉是机器人感知的重要组成部分,但其目前的技术水平难以帮助机器人有效地感知周围的三维世界。随着近几年深度学习的发展,计算机视觉领域取得了巨大的发展,尤其是在2D视觉领域,2D目标的检测和分类的准确率得到了巨大的提升,但3D目标的检测并没有巨大的提升。因此,本文针对机器人目前三维感知的困难,通过参考深度学习在2D视觉上的突破,将其引入到3D视觉上来,提出了3D-MRAI算法,用于解决对3D目标的检测以及位姿的估计。

  深度信息的质量对3D目标检测以及位姿估计的准确率和精度都有至关重要的影响,因此为了获取高质量的深度信息,本文针对现有RGB-D相机的缺点,提出了对偶RGB-D相机结构,通过组合两个低价的RGB-D相机来获取高质量的深度信息,提高了相机获取深度图的填充率并且增强了深度信息的鲁棒性。

  为了能够在RGB-D图中检测出目标物体的种类以及位姿,本文提出的3D-MRAI算法分为两步,第一步在相机拍摄的三维点云中分割出目标物体点云;第二步通过点云匹配算法求解出目标的位姿。为了分割出目标物体点云,本文基于2D目标检测中的Faster R-CNN和Mask R-CNN两个算法,提出了3D Faster R-CNN和3D Mask R-CNN算法,3D Faster R-CNN和3D Mask R-CNN通过将深度图变换为HHA图,有效地利用三维信息,并结合RGB图完成对目标物体的检测,并且为了应对目标物体的各种姿态,算法还引入了Spatial Transformer结构。3D Faster R-CNN和3D Mask R-CNN相比Faster R-CNN和Mask R-CNN充分利用三维信息,对检测一些纹理较少(Textureless)的物体有着更高的准确率。为了求解目标的位姿,本文通过匹配目标物体点云和目标物体3D模型来实现,为此基于4PCS算法提出了A4PCS-ICP点云匹配算法,通过在改进4PCS算法的基础上引入ICP算法提高了匹配精度。

  本文还将所提出的3D-MRAI算法实际应用到Bin-Picking问题上,设计了一个基于3D-MARI算法的随机分拣视觉系统,所设计的系统在实验中达到了100\%的抓取成功率,并且算法的运算时间也完全满足实际应用。
\end{cabstract}

\ckeywords{RGB-D,目标检测,位姿估计,随机分拣}


\begin{eabstract}
3D vision is an important part of robot perception, but the technology of 3D vision currently is hard to help robots effectively perceive the surrounding 3D world. With the development of deep learning in recent years, tremendous development has been made in the field of computer vision. Especially in the field of 2D vision, the accuracy of 2D object detection and classification has been greatly improved, but the detection of 3D objects has not been huge Enhance. Therefore, for the difficulty of current 3D robot perception. This paper proposed a new algorithm 3D-MRAI, which introduced deep learning into 3D vision based on the breakthrough of deep learning in 2D vision. This algorithm is proposed to solve the problem of 3D object detection and pose estimation .

High-quality depth map has a great influence on the results of 3D object detection and pose estimation. To acquire high-quality depth map, a dual RGB-D camera structure is proposed to overcome the shortcomings of the existing RGB-D cameras. Dual RGB-D camera can obtain high-quality depth map by combining two low-cost RGB-D cameras, which also increases the fill rate of depth map and enhances depth value in depth map.

To detect object and estimate pose in a given RGB-D frame, 3D-MRAI algorithm proposed this paper runs in two stage. The first stage is to get the point cloud of the object from the RGB-D frame. The second stage is to estimate the object pose by point cloud matching. To segement the point cloud of the object, 3D Faster R-CNN and 3D Mask R-CNN are proposed based on two algorithm in 2D object detection: Faster R-CNN and Mask R-CNN. 3D Faster R-CNN and 3D Mask R-CNN take full advantage of depth value by converting depth map into HHA frame and detect object by combining RGB and HHA. 3D Faster R-CNN and 3D Mask R-CNN also use Spatial Transformer to detect object in arbitrary pose.  3D Faster R-CNN and 3D Mask R-CNN have higher detection accuracy when detecting textureless objects, comparing to Faster R-CNN and Mask R-CNN. In order to estimate the pose of the target, we match the 3D model of the object to the point cloud of the object. For this purpose, a new point cloud matching algorithm call A4PCS-ICP is proposed. A4PCS-ICP algorithm has higher matching accuracy by combining ICP and modified 4PCS.

  This paper also applies the proposed 3D-MRAI algorithm to solve the Bin-Picking problem. A bin-picking vision system is designed based on 3D-MARI algorithm. The designed system achieves 100\% successful picking rate, and the system's response time also fully meets the application.

\end{eabstract}

\ekeywords{RGB-D, object detection, pose estimation, Bin-Picking}

%%% Local Variables:
%%% TeX-master: "../thesis.tex"
%%% End: