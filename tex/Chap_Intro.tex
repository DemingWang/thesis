%!TEX root = ../thesis.tex
\chapter{引言}
\label{chap:introduction}
随着现代科学技术的发展和人们生活质量的提高,对于工程结构的性能提出了越来越高的要求。例如:现代精密仪器、大型设备往往对于振动与位移有严格的限制;生命线工程结构,要求在大震和大灾作用下依然保有必要的功能,以为灾后救援与重建提供保障。20世纪中叶以来,尽管社会发展水平有了巨大的提高,然而由于灾害性作用而造成的损失却反而越来越大,这给结构工程学科带来了一系列新的挑战性课题。正是在这样的背景下,基于性能的设计思想开始浮出水面,并在近十年来引起了学者们强烈的兴趣。

......

“自然界只有一个,自然现象遵循着不依赖于人类意志的客观规律。然而,数理科学中却有着两套反映这些规律的体系:确定性描述和概率论描述。”(郝柏林,1997) 虽然概率论方法的发展引起了科学家和哲学家们关于自然本质的讨论,但是直到本世纪五十年代以前,两套方法在各自独立的领域内都得到了长足的发展。六十年代以来,由于本质非线性行为特别是混沌、分形等现象的发现和深入研究,随机方法的重要性得到了日益深刻的认识(Mandelbrot,1995)。人们发现,在确定性非线性系统的长期演化行为中会出现与随机行为不能加以区别的现象。而采用概率密度演化描述的方法却能很好地描述其演化密度的长期行为(Prigogine, 1996;郝柏林,1997)。
\section{随机结构分析现状}

\subsection{线性随机结构分析}
经过三十余年的发展,线性随机结构在静力与动力分析方面的分析方法均已
趋于成熟。早期在物理学研究中使用的随机模拟方法于20世纪70年代初期引入随机结构分析以来,已经成为检验各种随机结构分析方法的基本手段。基于随机摄动展开的随机结构静力分析与动力分析也已于20世纪80年代基本完善(李杰,1996)。